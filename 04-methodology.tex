\chapter{Methodology}
\label{chap:methodology}

Choosing which Software Development Life Cycle (SDLC) methodology was a key decision at the beginning of the project; the methodology influences the route that development took during the project's lifetime. The Iterative Development Model has been chosen alongside the use of other key project management methods in other team-focused methodologies, such as Agile \see{chap:pm}.

\section{An Iterative Development Methodology}
\label{methodology:chosen}

There was an initial dissonance when deciding between the Iterative and Incremental models. They are similar, but the Incremental model delivers a portion of the software at a time, whereas the Iterative model delivers the entire software at the end of each iteration. Often, questions were raised as to which model was being used, despite having initially chosen the Incremental model. 

The model was used for this project since the Waterfall Model cannot precisely and completely describe the real software development life cycle (\cite{dapeng_liu_case_2011}). Each iteration represents a full software life cycle vaguely following the waterfall methodology's structure: Requirements analysis, Design, Development, Testing, and Release. The project's requirements were also not fully known at the beginning of the project and were, therefore, subject to change. 

The Iterative development model provided more flexibility during development with the ability to accommodate changes to requirements. The model breaks larger tasks into smaller, more achievable sub-tasks/iterations. Each user story was broken down into sub-tasks (System Requirements) \see{chap:requirements} with each sub-task having its deadline to ensure changes were made within the iteration. Kanban boards were used to track the progress of all tasks \see{pm:kanban}.

One key downside of the iterative model is when changes are merged between iterations. This introduces a potential discontinuity of design purpose where the user interface and programming interfaces become discontinuous between iterations (\cite{dapeng_liu_case_2011}). To mitigate this, a consistent programming interface was implemented to ensure the development of easy-to-read source code.

\section{Conclusion}

To conclude, the Iterative Model was a pivotal decision that shaped the project's development approach. It was complemented by various project management strategies \see{chap:pm} to ensure the project was managed effectively. It's critical to understand the synergy between the chosen methodology and project management techniques, as they are both invaluable to the project's success.
