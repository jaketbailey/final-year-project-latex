\chapter{Methodology}
\label{chap:methodology}

Choosing which Software Development Life Cycle (SDLC) methodology is a key decision at the beginning of any software development project; the methodology demonstrates the expected route that development will take during the project's lifetime. I have decided to use the Iterative methodology whilst integrating key Project Management methods in other team-focused methodologies, such as Agile \see{chap:pm}.

\section{An Iterative Development Methodology}
\label{methodology:chosen}

I have chosen the Iterative Development Model for this project since the Waterfall Model cannot precisely and completely describe the real software development life cycle (\cite{dapeng_liu_case_2011}).
Each iteration will represent a full software life cycle vaguely following the waterfall methodology's structure: Requirements analysis, Design, Development, Testing, and Release. Iterative development allows for more flexibility during the software development process. It is feasible for a solo-development project as it does not require collaboration with other team members as the Agile methodology does.

The Iterative model breaks larger tasks into smaller, more achievable sub-tasks/iterations. Each task can be broken down into all or some of the stages mentioned earlier. Therefore, if some stages are unnecessary for an iteration, they don't need to be followed. Development of the iterations can be managed using a Kanban board. It can also manage progress in completing this document. To see the added benefits of using Kanban, see Section 4.2.

There are some downsides to using the Iterative model, with one key failure of the model being merging changes between iterations. This downfall of the model introduces a discontinuity of design purpose where the user interface and programming interfaces become discontinuous between iterations (\cite{dapeng_liu_case_2011}). To mitigate this issue, a consistent programming interface be implemented to aid in developing easy-to-read source code throughout the development of iterations.