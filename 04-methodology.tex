\chapter{Methodology}
\label{chap:methodology}

Choosing which Software Development Life Cycle (SDLC) methodology was a key decision at the beginning of the project; the methodology demonstrates the route that development took during the project's lifetime. The Iterative Development Model has been chosen alongside the use of other key project management methods in other team-focused methodologies, such as Agile \see{chap:pm}.

\section{An Iterative Development Methodology}
\label{methodology:chosen}

The project will use the Iterative Development Model for this project since the Waterfall Model cannot precisely and completely describe the real software development life cycle (\cite{dapeng_liu_case_2011}).
Each iteration represents a full software life cycle vaguely following the waterfall methodology's structure: Requirements analysis, Design, Development, Testing, and Release. Iterative development allows for more flexibility software development process, and therefore, is feasible for a solo development project due to the lack of collaboration required with other team members as the Agile methodology does.

The Iterative model breaks larger tasks into smaller, more achievable sub-tasks/iterations. Each task is be broken down into all or some of the stages mentioned earlier. Therefore, if some stages are unnecessary for an iteration, they don't need to be followed. Both development and report writing are managed using a Kanban board to track the progress and status of each task. To see the added benefits of using Kanban, see Section 4.2.

One key downside of the iterative model is when changes are merged between iterations. This introduces a potential discontinuity of design purpose where the user interface and programming interfaces become discontinuous between iterations (\cite{dapeng_liu_case_2011}). To mitigate this, a consistent programming interface was implemented to ensure the development of easy-to-read source code.