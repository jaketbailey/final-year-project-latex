\chapter{Evaluation}
\label{chap:evaluation}

This chapter evaluates the project as a whole with a focus on the development process, and whether the project meets the requirements set out in \autorefp{chap:requirements}. The chapter will also discuss the limitations of the project and potential future work.

\section{Project Timeline and Management}
\label{evaluation:timeline-management}

During the early stages of the academic year, starting a project was exciting and was progressing well. The problem would prove challenging over time, yet rewarding when understanding the fundamentals which build route planning systems. Time-management was key to ensure the project progressed at a steady pace despite the pressing deadlines of other modules. It was key to set out a plan and determine theoretical deadlines for development and writing of the report. Due to the rigorous time management and planning, the project swiftly began to move ahead of schedule, resulting in the initial Gantt chart \see{fig:initial-gantt} being revised to adjust for the rapid progress \see{fig:final-gantt}.

The first development iteration began around November before the designs were complete and requirements gathering phase had begun. This lead to the first iteration of the system being developed blindly, with no clear direction. Therefore, the first iteration merely set up the basic structure of the system, leading to development choices which would later not align with the requirements \see{chap:requirements}. Furthermore, because development begun before designs were also complete, the iteration didn't accommodate for the design decisions made after the iteration was complete \see{chap:design}. This lead to a significant amount of refactoring and rework in the second iteration, which could have been avoided if the development was started after the designs were complete.

The development methodology had also changed after iteration 1 was complete, moving from an incremental to an iterative approach. The incremental approach was not feasible due to the complexity of the project, it was more beneficial to progressively build and improve all aspects of the system in iterations. This allowed for the system to be built in a more modular fashion, allowing for easier integration of new features and changes. This proved beneficial as additions and changes were made to the React components and the backend throughout development as the requirements were formed and the system was improved and refined.

The final gantt chart demonstrates the ongoing process of the project, with dedicated development period catered around the progress of the report. The core development periods were between November and December, then January to the start of semester two. Approximately, development consisted of two months, on average twenty to twenty-five hours each week. Breaking up the development in such a way enabled for both development and report writing to be progressing effectively without one hindering the other.

The project overall was managed more effectively than originally planned, running ahead of schedule throughout the academic year. Having extra time allowed for refinements and improvements without the pressure of an immediate deadline. Constant feedback from the client yielded a more developed understanding of the requirements, the projects direction and suggestions for new requirements. The requirements originated as user stories, broken down into smaller, system requirements. Doing so made the use of GitHub projects and issues more effective with the ability to visualise project progress, further aiding to an iterative agile development process.

To summarise, the project was managed effectively, with the exception of the early start of iteration 1. The time spent on both development and report writing was balanced well around other modules and commitments, limiting the chance of burnout. The primary issue was the early stages where development decisions were made without designs and requirements which had a knock-on effect in later project stages when integrating more complex features with pre-existing code. It is critical to ensure the project is well planned and development doesn't begin until the requirements and designs are complete to limit the chance of future rework and refactoring.

\begin{figure}[h!]
    \centering
    \includegraphics[width=1\linewidth]{figures/Old FYP Gantt - Timeline 1.pdf}
    \caption{Initial Gantt Chart}
    \label{fig:initial-gantt}
\end{figure}

\begin{figure}[h!]
    \centering
    \includegraphics[width=1\linewidth]{figures/Actual FYP Gantt.pdf}
    \caption{Final Gantt Chart}
    \label{fig:final-gantt}
\end{figure}

\clearpage
\section{Evaluating Requirements}
\label{evaluation:requirements}

For this project, assessing its success based solely on the fulfilment of the defined requirements might not provide a comprehensive evaluation. All requirements were established ahead of the three development iterations, with a revision made after primary research was completed \see{fig:research-results}. User stories were created and broken down into smaller system requirements, which were assigned to the development iterations based on priority ordering via the MoSCoW method \see{chap:requirements}. Further analysis was required to determine is all requirements were met and if they combine to meet all project objectives. For all discussion surrounding the project's objectives \see{chap:reflection-and-conclusion}.

Overall, 47 of 51 system requirements were met \see{evaluatedrq}. The four requirements not met was due to the prioritisation of other requirements, the missed requirements were not critical to the system's functionality. The primary reason these were not met was due to the focus on \hyperref[SR:50]{SR50} at the end of development which was a complex but ideal feature to implement. Therefore, some of the initial requirements were not met, should there have been extra development time, these would be the first to be implemented.

The requirements missed were primary from 'User Story 05' \see{tab:user-story-05} due to the requirements' prioritisation ranking as 'Could' with the MoSCoW method. These focused on integrating weather data into the route planning algorithm. Whilst consideration occurred for extending development and implementing, there was no such weather data API which would serve effectively for this specific purpose. This was primarily due to limited funds and the complexity of the feature.

All requirements met can be evidenced by the work undertaken within the 'Implementation' chapter of this report \see{chap:implementation}. The routing machine provides users with core routing functionality and customisability (\hyperref[tab:user-story-02]{User Story 02}), whilst provides the user with the ability to export and share routes locally or to external services such as Garmin, Strava and Google Drive (\hyperref[tab:user-story-03]{User Story 03}, \hyperref[tab:user-story-04]{User Story 04}) \see{fig:garmin-connect}. Furthermore, the map component enables users to view and manipulate the route (\hyperref[tab:user-story-06]{User Story 06}) \see{fig:route-import}.

Many requirements were reliant on others to be built first, therefore the MoSCoW method was both used to determine the prioritisation based on user feedback with consideration of requirements' dependencies on other requirements. This enabled a fully functional, well-rounded system to be developed which was developed in a logical order, with a focus on modularity. Reflections on the requirements and objectives can be found in the conclusion \see{reflection-and-conclusion:outcomes/objectives-met}

\begingroup
\setlength{\tabcolsep}{10pt} % Default value: 6pt
\renewcommand{\arraystretch}{1.5} % Default value: 1
\begin{table}[!htb]
\caption{Requirements Evaluation}
\label{evaluatedrq}
\small
    \begin{tabularx}{\textwidth}{ p{1cm} p{11cm} p{1cm} }
        \hline
        ID & Description & Met? \\ 
        \hline
        & \textbf{\hyperref[tab:user-story-01]{User Story 01}} \\
        \hyperref[SR:1]{SR1} & The system must provide a route configuration panel. & Y\\
        \hyperref[SR:2]{SR2} & The route configuration page must provide a starting and destination location input field. & Y\\
        \hyperref[SR:3]{SR3} & The route configuration page should suggest accurate geolocations based on the location inputs.  & Y\\
        \hyperref[SR:4]{SR4} & The route configuration page must determine the geolocation based on the user input. & Y\\
        \hyperref[SR:5]{SR5} & The route configuration page must plan the route once two or more locations are input. & Y\\
        \hline
        & \textbf{\hyperref[tab:user-story-02]{User Story 02}}  \\
        \hyperref[SR:6]{SR6} & The system must provide an overlay window to allow the user to update routing preferences. & Y\\
        \hyperref[SR:7]{SR7} & The update preferences overlay must provide options to 'avoid' along the route. & Y\\
        \hyperref[SR:8]{SR8} & The update preferences overlay must provide a 'via' user input field. & Y\\ 
        \hyperref[SR:9]{SR9} & The update preferences overlay must provide a 'leave time' user input field. & Y\\ 
        \hyperref[SR:10]{SR10} & The update preferences overlay must provide a 'arrive time' user input field. & Y\\ 
        \hyperref[SR:11]{SR11} & The update preferences overlay must provide a 'round trip' user input field. & Y\\ 
        \hline
        & \textbf{\hyperref[tab:user-story-03]{User Story 03}}  \\
        \hyperref[SR:12]{SR12} & The system must provide an option to export the planned route. & Y \\
        \hyperref[SR:13]{SR13} & The system must provide an export feature to export the route to the 'GPX' file format. & Y\\
        \hyperref[SR:14]{SR14} & The system must provide an export feature to export the route to the 'GeoJSON' file format. & Y\\ 
        \hyperref[SR:15]{SR15} & The system must provide an export online (to Google Drive, OneDrive and/or other cloud services) & Y\\
        \hline
        & \textbf{\hyperref[tab:user-story-04]{User Story 04}}  \\
        \hyperref[SR:16]{SR16} & The system must provide a share functionality overlay. & Y \\
        \hyperref[SR:17]{SR17} & The share overlay must provide an option to share direct over email. & Y\\
        \hyperref[SR:18]{SR18} & The system must provide an option to share the route direct to Strava. & Y\\ 
        \hline
    \end{tabularx}
\end{table}
\clearpage

\begin{table}[!htb]
    \ContinuedFloat
    \caption{Requirements Evaluation Continued}
    \label{evaluatedrqextended}
    \small
    \begin{tabularx}{\textwidth}{ p{1cm} p{11cm} p{1cm} }
        \hline
        ID & Description & Met? \\ 
        \hline
        & \textbf{\hyperref[tab:user-story-05]{User Story 05}} \\
        \hyperref[SR:19]{SR19} & The system must provide the user with a weather condition overlay. & Y \\
        \hyperref[SR:20]{SR20} & The weather condition overlay must provide the user with the weather for the current day. & Y\\
        \hyperref[SR:21]{SR21} & The weather condition overlay must provide the user with the weather for the next week. & Y\\
        \hyperref[SR:22]{SR22} & The weather condition overlay must provide the user with the option to enable weather conditions in the route planning algorithm. & N\\ 
        \hyperref[SR:23]{SR23} & The weather condition overlay must provide the user with suggestions on the best days to cycle. & N\\
        \hyperref[SR:24]{SR24} & An option to include weather in route planning should be provided, ensuring the user enters the planned day to ride & N\\ 
        \hline
        & \textbf{\hyperref[tab:user-story-06]{User Story 06}}  \\
        \hyperref[SR:25]{SR25} & The system must provide the user with an interactive map to display the planned route. & Y \\
        \hyperref[SR:26]{SR26} & The interactive map must allow the user to zoom into parts of the planned route. & Y\\
        \hyperref[SR:27]{SR27} & The interactive map must allow the user to select parts of the route and receive detailed information about that subsection of the route. & Y\\
        \hyperref[SR:28]{SR28} & The interactive map must allow the user to select and drag the planned route to modify its path. & Y\\ 
        \hyperref[SR:29]{SR29} & The system must display an elevation graph for the planned route beneath the interactive map. & Y\\
        \hyperref[SR:30]{SR30} & The system must allow the user to measure chosen sections of the route & Y\\
        \hyperref[SR:31]{SR31} & The system must provide multiple map layers to give users the greater options when viewing the route & Y\\ 
        \hline
        & \textbf{\hyperref[tab:user-story-07]{User Story 07}}  \\
        \hyperref[SR:32]{SR32} & The system must provide a user input modal to input Hazard and Infrastructure Data. & Y \\
        \hyperref[SR:33]{SR33} & The hazard input modal must provide a Type drop-down menu based on the OSM Hazard Types. & Y\\
        \hyperref[SR:34]{SR34} & The hazard input modal must provide a date entry point to specify the date the hazard was seen. & Y\\
        \hyperref[SR:35]{SR35} & The hazard input modal must provide a submit button to add the hazard to the hazard index. & Y\\
        \hyperref[SR:36]{SR36} & The infrastructure input modal must provide a Type drop-down menu with different types of cycling/road infrastructure & Y\\
        \hyperref[SR:37]{SR37} & The infrastructure input modal must provide a date entry point to specify when the bad infrastructure was found & Y\\
        \hline
    \end{tabularx}
\end{table}
\clearpage

\begin{table}[!htb]
    \ContinuedFloat
    \caption{Requirements Evaluation Continued}
    \label{evaluatedrqextended2}
    \small
    \begin{tabularx}{\textwidth}{ p{1cm} p{11cm} p{1cm} }
        \hline
        ID & Description & Met? \\ 
        \hline
        & \textbf{\hyperref[tab:user-story-07]{User Story 07 Cont.}}  \\
        \hyperref[SR:38]{SR38} & The infrastructure input modal must provide an input box providing the user with the option to supply more detail & Y\\
        \hyperref[SR:39]{SR39} & Both Hazard and Infrastructure data should be displayed on the map, with an option to toggle on/off, and report errors & Y\\
        \hline
        & \textbf{\hyperref[tab:user-story-08]{User Story 08}}  \\
        \hyperref[SR:40]{SR40} & The system must provide a map layer to include key waypoints. & Y \\
        \hyperref[SR:41]{SR41} & The waypoint layer must provide locations of accommodation along the route. & Y\\
        \hyperref[SR:42]{SR42} & The waypoint layer must provide locations of tourist points along the route & Y\\
        \hyperref[SR:43]{SR43} & The waypoint layers must be able to be toggled on and off & Y\\
        \hyperref[SR:44]{SR44} & Each waypoint must be clickable to provide extra detail on each point & Y\\
        \hyperref[SR:45]{SR45} & Each waypoint must have a button to add stop along the route and the route will be re-plotted via the waypoint. & Y\\
        \hline
        & \textbf{\hyperref[tab:user-story-09]{User Story 09}}  \\
        \hyperref[SR:46]{SR46} & The system must provide an option within the route planning modal to select the rider type. & Y\\
        \hyperref[SR:47]{SR47} & The system must provide an option within the route planning modal to select the fitness level of the rider. & N\\
        \hline
        & \textbf{\hyperref[tab:user-story-10]{User Story 10}}  \\
        \hyperref[SR:48]{SR48} & The system must provide an import option for GPX files. & Y\\
        \hyperref[SR:49]{SR49} & The system must provide an import option for GeoJSON files. & Y\\
        \hline
        & \textbf{\hyperref[tab:user-story-11]{User Story 11}} \\
        \hyperref[SR:50]{SR50} & The system must provide an export to Garmin Routes. & Y\\
        \hyperref[SR:51]{SR51} & The system must provide an option to share to different social media platforms. & Y\\
        \hline
    \end{tabularx}
\end{table}
\endgroup

\section{Limitations}
\label{evaluation:limitations}

\section{Future Work}
\label{evaluation:future}

