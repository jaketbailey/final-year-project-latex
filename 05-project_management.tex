\chapter{Project Management}

\section{Methodology}

As stated throughout \see{chap:methodology}, the chosen methodology for this project is the Incremental SDLC Model. The project will be split into increments to break down larger tasks into smaller, more manageable subtasks. Each subtask will have its deadline to contribute to the changes within that increment. There will be regular incremental changes during the development and implementation of the artefact. All tasks to be completed will be managed with GitHub projects utilising its Kanban, Table and Milestones functionality, mentioned in Section 4.2.

\section{GitHub}

\subsection{Kanban Board and Projects}
\label{pm:kanban}

From the beginning of the project, a GitHub project was created alongside the GitHub Repository to manage all tasks to be completed for this document and the development of the artefact. All tasks to be completed were added to the backlog at the beginning of development and allocated a milestone and label to categorise each task.

Milestones were created for each Increment of the prototype and for each section of this document to be completed. This would allow tasks to be assigned to each milestone to better manage which task belongs to which overall objective. Each task would then be assigned a 'To Do' status when selected for development/writing and would progress throughout the other Kanban Board stages until it is marked as 'Done' and closed.

An extra status has been added to the Kanban board with the name 'In Review'; tasks placed in this status were awaiting guidance from the project supervisor \see{pm:supervisor_meetings}.

\subsection{Version Control}
\label{pm:version_control}

GitHub will host the project source code with Git Software Version Management (SVM). Using GitHub and Git enables branches to be created in the repository to manage code changes and link each change to a pull request which enables incremental development. Branches will allow for branch-level testing and pull requests for automated merges once the increment has been completed. These merges will also throw merge conflicts if issues in the code arise between increments, ensuring the code base remains accurate and forcing the conflict to be resolved before changes are made. Commits are also highly-valuable during development as they allow code changes to be managed and reversed if needed. 

\section{Supervisor Meetings}
\label{pm:supervisor_meetings}

Meetings will be scheduled on a weekly or fortnightly basis; my supervisor allocates a range of slots each week where I can book a time which works for me. The aim is to have all meetings face-to-face where possible. However, if the supervisor or I become unwell, we will conduct a video conference where possible. If the supervisor has planned leave, or I will be away for some time, we will communicate ahead of time to devise a plan for the project while one or the other is away and unable to meet regularly. Doing so will further allow me to plan my workload ahead of time and effectively keep track of the progress throughout the project.

[[INSERT LIST OF MEETINGS]]