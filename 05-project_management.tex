\chapter{Project Management}
\label{chap:pm}

\section{Methodology}

As stated throughout \see{chap:methodology}, the Iterative SDLC model was chosen for this project. The project will be divided into iterations to break down tasks into manageable subtasks. Each subtask will have its deadline to make all changes for that iteration. There will be regular iterative changes during the development and implementation of the artefact. All tasks will be managed with GitHub projects Kanban, Table and Milestones functionality \see{pm:kanban}.

\section{GitHub}

\subsection{Kanban Board and Projects}
\label{pm:kanban}

From the beginning, a GitHub project was created with the code repository to manage all development and writing tasks to be completed. All tasks were initially added to the backlog and allocated a milestone and label to categorise each task.

Milestones were created for each iteration of the prototype and each section of this document. Each task would then be assigned a 'To Do' status when selected for development/writing and would progress throughout the other Kanban Board stages until it is marked as 'Done' and closed. This allowed for the effective management of subtasks on a per-milestone and status basis.

An 'In Review' status was added to the Kanban board for tasks that required guidance from the project supervisor \see{pm:supervisor_meetings}.

\subsection{Version Control}
\label{pm:version_control}

GitHub will host the source code with Git Software Version Management (SVM). Using GitHub and Git enables branches to be created in the repository to manage code changes and link each change to a pull request which enables effective iterative development. Branches allow for branch-level testing and pull requests for automated merges once the increment has been completed. These merges will also throw merge conflicts arise between iterations, ensuring the code base remains accurate, forcing the conflict to be resolved before changes are made. Commits are also highly valuable during development as they allow code changes to be managed and reversed if needed. 

\section{Supervisor Meetings}
\label{pm:supervisor_meetings}

With the project requiring regular client interaction, regular meetings were scheduled with the supervisor to monitor progress. In-person meetings were preferred, however, due to illness, video conferences were used as a backup. See the list of meetings below:

\begin{itemize}
    \item \textbf{Meeting 1 - 26th September 2023}
    \begin{itemize}
        \item Discussion of the initial project idea.
        \item Discussion of different approaches to the project.
        \item Discussion of the best way to work with the supervisor.
    \end{itemize}
    \item \textbf{Meeting 2 - 5th October 2023}
    \begin{itemize}
        \item Feedback on initial draft project initiation document.
        \item Outlining overall project objectives.
    \end{itemize}
    \item \textbf{Meeting 3 - 11th October 2023}
    \begin{itemize}
        \item Finalise and submit project initiation document.
        \item Discussion of ethics application.
        \item Discussion of potential primary research methods.
    \end{itemize}
    \item \textbf{Meeting 4 - 18th October 2023}
    \begin{itemize}
        \item Demo and feedback of initial research prototype application.
        \item Demo and feedback of initial basic artefact setup.
        \item Discussion of potential future features.
    \end{itemize}
    \item \textbf{Meeting 5 - 25th October 2023}
    \begin{itemize}
        \item Feedback on document progress so far, working towards literature review.
        \item Discussion on ways to rank and compare feedback from the research application.
    \end{itemize}
    \item \textbf{Meeting 6 - 30th October 2023}
    \begin{itemize}
        \item Feedback on literature review.
        \item Discussion of requirements using research feedback.
    \end{itemize}
    \item \textbf{Meeting 7 - 14th November 2023}
    \begin{itemize}
        \item Feedback on complete literature review.
        \item Feedback and discussion of requirements.
        \item Further discussion of potential future features on top of requirements.
    \end{itemize}
    \item \textbf{Meeting 8 - 29th November 2023}
    \begin{itemize}
        \item Requirements mostly finalised, feedback received.
        \item Discussion on the design of the artefact.
    \end{itemize}
    \item \textbf{Meeting 9 - 8th December 2023}
    \begin{itemize}
        \item Final feedback on project so far for 2023.
        \item Discussion of development/sprint plan for January 2024.
        \item Discussion of plan post-development.
    \end{itemize}
    \item \textbf{Meeting 10 - 10th January 2024}
    \begin{itemize}
        \item Feedback on development progress so far.
        \item Agreement made to delay meetings for 3-4 weeks to allow for development.
        \item Agreement on the plan for post-development (3-4 weeks).
    \end{itemize}
    \item \textbf{Meeting 11 - 9th February 2024}
    \begin{itemize}
        \item Feedback on development
        \item Discussion of end-goal, final artefact/document.
        \item Discussion of the plan for the project stages.
    \end{itemize}
    \item \textbf{Meeting 12 - 16th February 2024}
    \begin{itemize}
        \item Feedback on Design and Implementation chapters.
        \item Agreement on final end goal - 5 weeks from 16th February.
        \item Agreement to complete draft of final document within 2 weeks.
        \item Agreement to use final 3 weeks to review and improve.
        \item Discussion of the use of unplanned extra time (after 5 weeks) to improve artefact.
    \end{itemize}
\end{itemize}