\chapter{Reflection and Conclusion}
\label{chap:reflection-and-conclusion}

This chapter reflects on the project as a whole and concludes the report. It highlights both the positives and negatives of how the project was conducted and outlines the lessons learned across the course of the project.

\label{reflection-and-conclusion:what-went-well}
\section{What Went Well}

LaTeX has been a powerful tool for writing the report. It has drastically improved the report's quality and has made the process of writing the report much more enjoyable. One beneficial LaTeX feature was the ability to easily cross-reference figures, tables and sections of the report. Automating the bibliography in the specified APA 7th Edition format was also crucial. The use of LaTeX Workshop within VS Code also improved the efficiency of writing the report because editing was completed in a familiar environment. Zotero also paired perfectly with the LaTeX bibliography, allowing for automated management of citations and references with auto export of the bibliography to a .bib file.

Using Go with the Gin Web Framework has been a great experience. The language has a simple and understandable syntax, yet is powerful and efficient. The adoption of types on the backend was refreshing, allowing for a more robust and reliable codebase. Custom types were used to validate incoming data and ensure that all data was correctly formatted and typed which aided greatly in reducing the complexity of input validation. Gin has much functionality that was unused in the project, but is available for future development, and therefore has been a pivotal choice for scaling the project in the future. The expansion of knowledge in Go and Gin has been a great learning experience and has provided a solid foundation for future projects.

\label{reflection-and-conclusion:what-didnt-go-well}
\section{What Didn't Go Well}

The primary pain point during development was utilising React.js to build the application. Despite React providing all the critical functionality required for such an app, the project's approach could have been revised to provide a better understanding of how specific data sets were mwill managed within the artefact. Issues with passing state variables were found due to the order of different events, therefore workarounds were used which could be avoided in the future. Other issues faced were highlighted in the evaluation section \see{evaluation:limitations}.

\label{reflection-and-conclusion:outcomes/objectives-met}
\section{Were Outcomes and Objectives Met}

The project outcomes were met, with a report, and an artefact produced within the set time frame. All objectives were outlined in the introduction \see{intro:aimsandobjectives}. Three of four objectives were met, except integrating weather data into the route planning algorithm which has been expanded upon within the evaluation \see{evaluation:requirements}. The remaining objectives were met, however specific functionalities could have been implemented more effectively \see{evaluation:future}.

The system offers a way of route planning with information for cyclists and runners. An interactive OSM map visualises the route paired with a UI that enables users to specify their route preferences, providing a step-by-step route plan \see{fig:poi-route}. Further information is provided to the user with an elevation chart which further allows users to zoom and pan into route segments \see{fig:elevation-hover}.

The project acts as an initial exploration into a nuanced and multifaceted problem regarding the customisability and safety of route planning for cyclists and runners. Answers were found with vast amounts of understanding gained around the problem space. The project still however raises the following questions:
\begin{itemize}
    \item What constitutes a balance between safety and custom route planning?
    \item What is the best way to entice crowdsourcing of data to improve the safety of routes?
\end{itemize}

\label{reflection-and-conclusion:final-conclusion}
\section{Final Conclusion}

The development of a route-planning app for cycling holds great promise for public health. By making it easier for people to find safe and enjoyable routes, the app can promote regular exercise, which is crucial for preventing many health issues. Additionally, engaging in outdoor activities like cycling and running can help improve mental well-being by reducing stress and anxiety of the public. Enhancing accessibility and customisation features in future iterations of the app could amplify its positive impact on individuals' health and community well-being. 

To conclude, most objectives were met when developing the artefact, however, many areas could still be improved. The aim is to continue the future development of the artefact to encompass a wider range of users and provide a more comprehensive and customisable route planning experience. The project has been a great learning experience and has provided vital skills to improve the artefact and develop future projects.

