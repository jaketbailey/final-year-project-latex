\chapter{Reflection and Conclusion}
\label{chap:reflection-and-conclusion}

This chapter reflects on the project as a whole, and concludes the report. It will highlight both the positives and negatives of how the project was conducted. Ot a;sp demonstrates the lessons learned across the course of the project.

\label{reflection-and-conclusion:what-went-well}
\section{What Went Well}

LaTeX has proven to be a powerful tool for writing the report. It has allowed for a major improvement in the report's quality and has made the process of writing the report much more enjoyable. The key features of LaTeX that have been beneficial are the ability to easily cross-reference figures, tables and sections of the report, whilst automating the bibliography in the specified APA 7th Edition format. The use of LaTeX Workshop within VS Code also greatly improved the efficiency of writing the report as it enabled editing in an environment that was already familiar for programming the artefact. Zotero also paired perfectly with the LaTeX bibliography, allowing for automated management of citations and references with auto export of the bibliography to a .bib file.

Using the Go programming language with the Gin Web Framework has been a great experience. The language has a simple and understandable syntax, yet is powerful and efficient. The adoption of types on the backend was refreshing, allowing for a more robust and reliable codebase. Custom types were used to validate incoming data and ensure that all data was in the correct format and of the correct type which aided greatly reducing the complexity of input validation and error handling. Gin has an extensive set of functionality which was unused in the project, but is available for future development, and therefore has been a pivotal choice for scaling the project in the future. The expansion of knowledge in Go and Gin has been a great learning experience and has provided a solid foundation for future projects.

\label{reflection-and-conclusion:what-didnt-go-well}
\section{What Didn't Go Well}

The primary pain-point during development was utilising React.js to build the application. Despite React providing all the critical functionality required for such an app, the project's approach could have been revised to provide a better understanding of how specific data sets ar managed within the artefact. Issues with passing state variables were found due to the order of different events, therefore different workarounds were used which could be avoided in the future. Other issues faced were highlighted in the evaluation section \see{evaluation:limitations}.

\label{reflection-and-conclusion:outcomes/objectives-met}
\section{Were Outcomes and Objectives Met}

The project outcomes were met, a report, and an artefact were produced within the set time frame. All objectives were outlined in the introduction \see{intro:aimsandobjectives}. Three of four objectives were met, with the exception of integrating weather data into the route planning algorithm which has been expanded upon within the evaluation \see{evaluation:requirements}. The remaining objectives were met, however specific functionalities could have been implemented more effectively \see{evaluation:future}.

\label{reflection-and-conclusion:final-conclusion}
\section{Final Conclusion}
