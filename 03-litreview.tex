\chapter{Literature Review} 
\label{chap:litrev}

This chapter presents the analysis and review of relevant literature conducted throughout this project. The research comprises of literature surrounding different routing algorithms and their current usage within current route planner applications. It also explores other relevant technologies which will be utilised in developing the artefact, such as web frameworks and programming languages.

\section{Background}
\label{litrev:background}
This project has a high level of complexity. It utilises custom, user and system inputs into a data-driven route planning algorithm, displaying the output on an OpenStreetMap-based web application. 

This review begins with researching literature on different web technologies and programming languages to develop an understanding of what technologies were available to build the proposed system. Once an understanding of the technologies was developed, a review of current competing products was reviewed. This allowed for gaps in the current market to be identified, therefore developing an idea of requirements for the proposed system \see{litrev:competingproducts}. 

Furthermore, research into how cyclists consider different risk factors when planning a ride was key in understanding what routing algorithm was implemented to fit the project's requirements best. An in-depth review of existing literature was conducted to understand two primary risk factors for cyclists: cycling infrastructure \see{litrev:cyclinginfrastructure} and weather conditions \see{litrev:weatherconditions}. 

\section{Research Methods}
\label{litrev:researchmethod}

An initial exploration of sources and subject areas was conducted using Google Chrome to comprehensively understand the topic at hand. Certain regions of interest were also highlighted through crowd-sourcing ideas from knowledgeable individuals. After these areas were outlined, in-depth research was conducted primarily through Google Scholar and the University of Portsmouth EBSCO database. The Zotero Chrome plugin and app managed citations and bibliography items (\cite{noauthor_zotero_nodate}). All bibliography items have also been stored in a CSV file, utilising Zotero's export functionality; doing so ensures that all relevant sources can be re-visited at any time and are not lost after this project has concluded.

\section{Competing Products}
\label{litrev:competingproducts}

There are many different route planners available with a range of different features, some common between applications and others specific to one. Applications like Plotaroute.com (\cite{noauthor_free_nodate}), Komoot (\cite{noauthor_komoot_nodate}) and Google Maps (\cite{noauthor_google_nodate}) have some commonalities, however serve slightly different purposes \see{fig:solutionsandfeatures}.

\begin{figure}[h!]
    \centering
    \includegraphics[width=1\linewidth]{figures/current_apps.pdf}
    \caption{Table of current solutions and their included features}
    \label{fig:solutionsandfeatures}
\end{figure}

\subsection{Plotaroute.com}
\label{litrev:plotaroute}
Plotaroute.com (\cite{noauthor_free_nodate}), now referred to as Plotaroute, contains a wide range of features, including nearly all features highlighted in Figure 2.1 \see{fig:solutionsandfeatures}. The main shortfall of Plotaroute was identified as its UI (User Interface) rather than the features included. The UI of Plotaroute is very cluttered due to the number of features present in the application \see{fig:plotarouteui}. Unless the user is an expert and has used the application before, it is initially confusing what each part of the application does. Due to this, at first glace, it's unclear what type of route planner Plotaroute is, which will fundamentally affect a user's initial decision on whether or not to use the application.

\subsection{Komoot}
\label{litrev:komoot}
Komoot (\cite{noauthor_komoot_nodate}) offers a simpler-looking yet feature-rich application for planning and discovering routes \see{fig:komootui}. The application has a key focus on community, whereby a user doesn't necessarily need to plan a route, they can simply discover a route or even share a route of their own. This functionality allows the user to require minimal effort when planning location-specific rides, however doesn't offer discovery of longer routes, for example, Land's End to John O'Groats. When compared to Plotaroute, Komoot is clearly more user-friendly without sacrificing the core features needed by users. One primary setback with Komoot, however, is that some functionality for route planning is part of a paid-for service, therefore locking certain user bases out of some key desirable functionality.

\subsection{Google Maps}
\label{litrev:gmaps}

Google Maps (\cite{noauthor_google_nodate}) further reduces the specific functionality and offers a very simple, multi-functional route planning and location finding application \see{fig:gmapsui}. Google Maps is likely the most user-friendly out of all the applications, simply due to its simplicity and consistency across other Google applications (\cite{noauthor_material_nodate}). With this simplicity, however, most cycling-specific functionality is not present; the only option the user has when calculating a route is what the Google routing algorithm calculates with potentially a few alternate routes. Therefore, limiting how much the user can customise their route.

\label{fig:research-results}
\begin{figure}[h!]
    \centering
    \includegraphics[width=1\linewidth]{figures/current_apps_post_research.pdf}
    \caption{Features in priority order based on user feedback}
    \label{fig:features}
\end{figure}

\section{Risk Factors in Route Planning}
Risk-based cycling route planning requires extensive knowledge of the impact of cycling and transportation infrastructure currently in place. It is also critical to understand how other external factors impact the risk of a route on an ever-changing basis. Within this section, a range of risk factors were explored to understand how multiple risk factors can be implemented into route planning algorithms.

\subsection{Cycling Infrastructure}
\label{litrev:cyclinginfrastructure}
The cycling infrastructure along a route must be understood because it is common for cyclists to share the same infrastructure as motorised vehicles. However, a cyclist has no physical protection if a crash occurs (\cite{reynolds_impact_2009}). There is often purpose-built infrastructure for cyclists, whether bike lanes alongside shared roads or off-road bike paths and this segregated infrastructure can help improve the safety of a route for a cyclist. Furthermore, Hong states how investing in effective cycling infrastructure "mitigates the negative effects of poor weather conditions" (\cite{hong_can_2020}), which further demonstrates that good, known infrastructure is key to improving the physical and perceived safety of a route in a range of different weather conditions. 

Furthermore, crowd-sourced data from route planners, cyclists and fitness applications such as Strava Routes (\cite{noauthor_strava_nodate}) have been key in developing new infrastructure. Boettge states how the most accurate assessment of a cycle network would come directly from the cyclists who use the network (\cite{boettge_assessing_2017}). Cyclists who use the network are the most familiar with the quality of each route and how traffic conditions improve the safety of the route. Utilising the GPS information from route planners and fitness tracking applications alongside direct input from cyclists can help build new routes and improve pre-existing routes, therefore preventing injuries and high-risk situations by modifying transportation infrastructure (\cite{reynolds_impact_2009}).

Areas with little to no cycling infrastructure, such as busy roads and roundabouts, force cyclists to have a heightened attentiveness that other road users don't have to consider due to not only the physical danger but the cyclists' perceived danger (\cite{doorley_analysis_2015}). These risks should be considered within route planning to decrease the number of 'risk' areas along a cyclist's route whilst also giving local areas the incentive to make infrastructure modifications to decrease the number of 'high-risk' points. Therefore, in the long-term, it will mitigate the need for constant action by cyclists to ensure their safety, which will, in turn, influence individuals' decisions to cycle (\cite{reynolds_impact_2009}).

Hull and O'Holleran also demonstrate how cities with a high reputation among cyclists also have safer roads and more attractive infrastructure. The Netherlands scored relatively equally amongst all categories, in comparison to cities with less of a reputation and, therefore, a lower standard of cycling infrastructure \see{fig:bicycleinfrastructurescorestable} \see{fig:bicycleinfrastructurescores} (\cite{hull_bicycle_2014}). This supports how Reynolds et al. further illustrate how investing in cycling infrastructure will greatly incentivise individuals to cycle due to the decreased risk. 

\begin{figure}
    \centering
    \includegraphics[width=400px, keepaspectratio]{figures/bicycle_infrastructure_score_table.jpg}
    \caption{Comparison of the bicycle infrastructure scores (\cite{hull_bicycle_2014}).}
    \label{fig:bicycleinfrastructurescorestable}
\end{figure}

\begin{figure}
    \centering
    \includegraphics{figures/bicycle_infrastructure_scores.jpg}
    \caption{Spider web diagram comparing the Bicycle Infrastructure Scores (\cite{hull_bicycle_2014}).}
    \label{fig:bicycleinfrastructurescores}
\end{figure}

\subsection{Weather Conditions}
\label{litrev:weatherconditions}g
Weather conditions will also have a pivotal effect on how a route planner will calculate the safest route for a cyclist. Following on from Cycling Infrastructure \see{litrev:cyclinginfrastructure}, it is demonstrated how a lack of good infrastructure goes hand-in-hand in creating an unsafe route alongside the weather. To ensure the safety of cyclists, all routes and road surfaces must be maintained to withstand different weather conditions (\cite{shoman_evaluation_2023}).

The weather also impacts a cyclist's likelihood to ride; Flynn states that cyclists `were nearly twice as likely to commute by bicycle when there was no morning precipitation' (\cite{flynn_weather_2012}). It is clear that even minor changes in the weather can drastically affect a cyclist's decision to ride, further demonstrating how vital the perceived safety of cycling is in deciding whether to ride. 

Contrasting this, Hull and O'Holleran state that the main environmental barriers included too much traffic, too many hills, no bike lanes/trails, no safe place to cycle and badly maintained streets (\cite{hull_bicycle_2014}). Therefore suggests that the weather should have a minimal impact on a cyclist's decision to ride if the infrastructure is sufficient. Despite the findings of Hull and O'Holleran, it seems to be a common finding that the perceived safety of cycling, both in regard to the changing weather conditions and cycling infrastructure, is the primary factor in choosing cycling over an alternative method of transport. Miranda-Moreno and Nosal have shown how when infrastructure is implemented, there is generally an increase in total bicycle usage and diversion of cyclist flows away from roads to purpose-built infrastructure even in less ideal weather conditions (\cite{miranda-moreno_weather_2011}).

\section{Conclusion}
\label{litrev:conclusion}

To conclude, route planning with different customisable preferences has been implemented by a range of different existing organisations; however, focusing on a risk-based routing approach has not been addressed by these existing solutions. Utilising pre-existing routing algorithms such as Open Route Service (\cite{noauthor_openrouteservice_nodate}) or Open Source Routing Machine (\cite{noauthor_project_nodate}) and integrating custom, weather and infrastructure data alongside the usual user-preferences has not been implemented within existing solutions. Therefore, this enables a unique system to be developed whereby crowd-sourced infrastructure data alongside weather data provided by OpenWeatherMap combined form a risk index utilised in a customised routing algorithm. %(\cite{noauthor_urrent_nodate})%

Furthermore, in order to develop this system, React.js was chosen to develop the front end and Go for the back end. Next.js was initially considered for the front-end. However, it was later found that Next's server-side rendering was not supported by Leaflet; used for Mapping with OpenCycleMap; (\cite{noauthor_leaflet_nodate};\cite{noauthor_opencyclemaporg_nodate}) due to it requiring direct interaction with the DOM. Go with the Gin Web Framework (\cite{noauthor_gin_nodate}) was chosen to develop the API and back-end due to its increased performance benefits over alternative languages such as Node.js with Express.js.