\chapter{Introduction}
\label{chap:intro}

\section{Overview}
\label{intro:overview}

Route planning is essential to all cyclists, it's important to know where they are going and what to expect on their journey. A casual rider may wish to find the quickest and safest route to their destination. In contrast, a professional rider would want to find the most challenging route, not the quickest.

The proposed solution is a web application enabling cyclists to plan a route ahead of their ride and choose which criteria they want to affect the routing algorithm and, therefore, the final route. Some examples of these criteria are accommodation, road type and elevation. Other criteria will be used in determining the route, such as weather conditions and hazards along the route.

\section{Aims and Objectives}
\label{intro:aimsandobjectives}

The overall project aim is to develop a prototype which can be expanded in the future to plan a range of routes for cyclists with a many data-driven criteria to cater to different user needs. Enabling customisable criteria for users enables them to understand the decisions made when planing their route, increasing a user's incentive to begin cycling.

The objectives of the project are:
\begin{itemize}
    \item Provide a way of planning a route for cyclists and provide key information about the route.
    \item Visualise the route on a map and provide a way to interact with the route.
    \item Allow the user to customise the route planning algorithm to cater to their needs.
    \item Integrate weather data into the route planning algorithm.
\end{itemize}

\section{Proposed Solution}
\label{intro:proposedsolution}

The solution is a prototype web application designed for cyclists of all levels. It will plan routes based on a range of data-driven, user-customisable and fixed criteria through a user-friendly, modern user interface (UI). Furthermore, the solution will provide information on the planned route, such as elevation and the average completion time.

\section{Scope}
\label{intro:scope}

There is a large scope that could be attributed to this project whereby the application could be developed to consider many different sports that could benefit from route planning. These sports could be, running, walking and mountain cycling, which would increase the scope due to the number of factors to consider for these sports. This would expand the current scope of the project to a great scale.

To mitigate the risk of the scope becoming unfeasible, it's vital to focus on the key functionality planned for the application. Therefore, the decision has been made to narrow the scope and focus explicitly on catering the application for road cyclists. Focusing on this sport will ensure focus on the key functionality to provide effective route planning on roads catering to the user's needs.

\section{Deliverables}
\label{intro:deliverables}

This project will consist of two deliverables:
\begin{itemize}
    \item The project report
    \item The artefact (web-based bicycle route planner)
\end{itemize}

\section{Resources and Facilities used}
\label{intro:resourcesandfacilities}

This section demonstrates all resources and facilities used to complete this project. It includes prototype development and writing this report; the list should aid anyone wishing to re-create this project. Some decisions on tools used are personal; for instance, the text editor, which is down to preference. On the other hand, the programming languages chosen are specific to the features and performance benefits they provide in developing such applications.

\begin{itemize}
    \item Applications/websites used to complete research and literature review
    \begin{itemize}
        \item Google Scholar
        \item EBSCO Database
    \end{itemize}
    \item Programming Languages
    \begin{itemize}
        \item Javascript
        \item HTML5
        \item CSS
        \item Go
        \item Structured Query Language (SQL)
    \end{itemize}
    \item Text Editor
    \begin{itemize}
        \item Visual Studio Code
    \end{itemize}
    \item Database Management System (DBMS)
    \begin{itemize}
        \item PostgreSQL
        \item PostGIS
    \end{itemize}
    \item Wireframes and Prototyping
    \begin{itemize}
        \item Marvel
    \end{itemize}
    \item Libraries
    \begin{itemize}
        \item React.js
        \item Vite
        \item Gin
        \item OpenRouteService (ORM)
        \item OpenStreetMap (OSM)
        \item Leaflet
        \item Leaflet Routing Machine
    \end{itemize}
    \item Testing
    \begin{itemize}
        \item Jest
        \item Go Test Framework
    \end{itemize}
    \item Browser
    \begin{itemize}
        \item Google Chromium
    \end{itemize}
    \item Project Management and Version Control
    \begin{itemize}
        \item GitHub
        \item GitHub Interactive Kanban Board
        \item GitHub Issues
        \item Git
    \end{itemize}
\end{itemize}

\section{Risk Assessment}
\label{intro:riskassessment}

It is vital to understand the potential risks at the beginning of the project to establish a way to mitigate those risks should they occur. The risks have been identified when considering this project \see{tab:risk-assessment}.

\begin{center}
\begin{longtable}{ p{2cm} p{2.25cm} p{2cm} 
p{1.5cm} p{5cm} } 
\caption{Project Risk Assessment} \\
\hline
\textbf{Type} & \textbf{Risk} & \textbf{Likelihood} & \textbf{Severity} & \textbf{Description and Mitigation}\\
\hline
    Personal &  Illness & Low & Medium & There is the chance that I may fall ill unexpectedly during the project. To mitigate the risk of this visit a doctor if unwell and allocate time for rest around university work.\\ \cline{2-5}
    & Supervisor illness & Medium & Medium &  The supervisor becoming unwell in any instance could impact the completion of tasks due to the lack of technical guidance and may delay the project’s timeline. If this occurs, we will conduct online meetings and plan time in preparation for the potential delay.\\
    \hline
    Management & Changing requirements & Medium & Medium &  During the project, the requirements initially suggested may change. To mitigate this, communicate regularly with the supervisor to discuss the feasibility of the current requirements to prevent any future changes.\\\cline{2-5}
    & Time Availability & Medium & Medium &  The time to dedicate to the project may become more limited as I have more work to complete for other modules. To prevent this, plan tasks ahead of time to consider potential future setbacks.\\
    \hline
    Technical & Data loss/corruption & Low & High &  Data loss would mean the prototype must be developed again from scratch. To mitigate this issue, create regular backups on GitHub and locally in case either fails.\\\cline{2-5}
    & Hardware failure & Low & High &  PC/Server/Laptop being used for development/hosting crashes/hardware gets damaged. Ensure multiple devices are being used so development can continue even whilst one device is down.\\\cline{2-5}
    & Documentation availability & Low & Medium &  Documentation for frameworks, languages or libraries is unavailable. There is a range of readily available online for all the required technologies.\\
    \hline
\label{tab:risk-assessment}
\end{longtable}
\end{center}

\clearpage
\section{Legal, Ethical, Social and Professional issues}
\label{intro:legal...issues}

The key legislation to consider for this project is the Data Protection Act 2018 (DPA 2018). The project should not store personal information. However, the user’s current location will be requested upon the launch of the application; when the application no longer uses this data, it will be deleted and re-requested when the user enters the application again. Future iterations could implement accounts, storing a small amount of sensitive user information to include more features. However, the submitted artefact shouldn’t contain this data; regardless of this fact, it will be ensured the artefact abides by all principles of the DPA 2018 due to it handling the current location data of the user.

One social issue that could arise is that the artefact may entice more public members to start cycling more frequently; whilst this result will be a great incentive for protecting the environment, some road users are cautious, with many cyclists riding unsafely on the roads. The artefact will push users to ride safely and abide by all road safety laws, just as vehicles do; there will also be the option only to use cycle routes/lanes when plotting a route to ensure those cyclists who are less comfortable on roads still feel safe on the routes planned by the artefact.