\chapter{Introduction}
\label{chap:intro}

\section{Overview}
\label{intro:overview}

Route planning is key for cyclists and offers insight into their destination and route expectations. A casual rider may wish to find the quickest and safest route to their destination. In contrast, a professional rider would want to find the most challenging route, not the quickest.

The proposed solution is a web application enabling cyclists to customise routing criteria and plan a route ahead of their ride. The solution will aim to incentivise more people to begin cycling, as well as providing a tool for professional cyclists to plan their routes. Furthermore, the solution should help reduce the number of cars on the road, therefore supporting the UK government's goal of reducing carbon emissions by 78\% by 2035 (\cite{govuk_net_2022}) and increasing the public's overall physical and mental health.

\section{Aims and Objectives}
\label{intro:aimsandobjectives}

The main aim is to develop an extensible prototype to plan routes for cyclists with many data-driven criteria to cater to different user needs. Enabling customisable criteria for users enables them to understand the route planning decisions made, increasing a user's incentive to begin cycling.
 
The objectives of the project are:
\begin{itemize}
    \item Provide a way of planning a route for cyclists and provide key information about the route.
    \item Visualise the route on a map and provide a way to interact with the route.
    \item Allow the user to customise the route planning algorithm to cater to their needs.
    \item Integrate weather data into the route planning algorithm.
\end{itemize}

\section{Proposed Solution}
\label{intro:proposedsolution}

The solution is a prototype web application designed for cyclists of all levels. It will plan routes based on a range of criteria through a user-friendly, modern user interface (UI). Furthermore, the solution will provide information on the planned route, such as elevation and the average completion time.

\section{Scope}
\label{intro:scope}

The potential scope for the project is large whereby the application could be developed to include many custom features which may detract from the initial aims. To mitigate the risk of the scope becoming unfeasible, it's vital to focus on the key functionality planned for the application, therefore, the decision has been made to develop a modular system that allows for future functionality to be added without expanding the current scope of the artefact. Limiting the scope will ensure the development of the key route planning functionality without preventing future expansion.

\section{Deliverables}
\label{intro:deliverables}

This project will consist of two deliverables:
\begin{itemize}
    \item The project report
    \item The artefact (web-based bicycle route planner)
\end{itemize}

\section{Resources and Facilities used}
\label{intro:resourcesandfacilities}

This section demonstrates all resources and facilities used to complete this project. It includes prototype development and writing this report; the list should aid anyone wishing to re-create this project. Some decisions on tools used are personal; for instance, the text editor, which is down to preference. On the other hand, the programming languages chosen are specific to the features and performance benefits they provide in developing such applications \see{appendix:tools_and_technologies}.

\section{Risk Assessment}
\label{intro:riskassessment}

Understanding and mitigating the potential risks at the beginning of the project is critical. The risks have been identified when considering this project \see{tab:risk-assessment}.

\clearpage
\begin{center}
\begin{longtable}{ p{2cm} p{2.25cm} p{2cm} 
p{1.5cm} p{5cm} } 
\caption{Project Risk Assessment} \\
\hline
\textbf{Type} & \textbf{Risk} & \textbf{Likelihood} & \textbf{Severity} & \textbf{Description and Mitigation}\\
\hline
    Personal &  Illness & Low & Medium & There is the chance that I may fall ill unexpectedly during the project. To mitigate the risk of this visit a doctor if unwell and allocate time for rest around university work.\\ \cline{2-5}
    & Supervisor illness & Medium & Medium &  The supervisor becoming unwell in any instance could impact the completion of tasks due to the lack of technical guidance and may delay the project’s timeline. If this occurs, we will conduct online meetings and plan time in preparation for the potential delay.\\
    \hline
    Management & Changing requirements & Medium & Medium &  During the project, the requirements initially suggested may change. To mitigate this, communicate regularly with the supervisor to discuss the feasibility of the current requirements to prevent any future changes.\\\cline{2-5}
    & Time Availability & Medium & Medium &  The time to dedicate to the project may become more limited as I have more work to complete for other modules. To prevent this, plan tasks ahead of time to consider potential future setbacks.\\
    \hline
    Technical & Data loss/corruption & Low & High &  Data loss would mean the prototype must be developed again from scratch. To mitigate this issue, create regular backups on GitHub and locally in case either fails.\\\cline{2-5}
    & Hardware failure & Low & High &  PC/Server/Laptop being used for development/hosting crashes/hardware gets damaged. Ensure multiple devices are being used so development can continue even whilst one device is down.\\\cline{2-5}
    & Documentation availability & Low & Medium &  Documentation for frameworks, languages or libraries is unavailable. There is a range of readily available online for all the required technologies.\\
    \hline
\label{tab:risk-assessment}
\end{longtable}
\end{center}

\clearpage
\section{Legal, Ethical, Social and Professional issues}
\label{intro:legal...issues}

The Data Protection Act (DPA 2018) was considered for this project. The project does not store personal information, however, the user’s approximate location is requested upon the launch of the artefact. When the artefact no longer uses this data, it will be deleted and re-requested when the user enters the application again. Future iterations could implement accounts, storing a small amount of sensitive user information to include more features, however, the submitted artefact doesn't contain this data; regardless of this fact, it will be ensured the artefact abides by all principles of the DPA 2018.

One social issue could be that the artefact may entice more people to start cycling more frequently. This result will be a great incentive for protecting the environment, however, some road users are cautious with many cyclists riding unsafely on the roads. The artefact will push users to ride safely and abide by all road safety laws, just as vehicles do; there will also be the option only to use cycle routes/lanes when plotting a route to ensure those cyclists who are less comfortable on roads still feel safe on the routes planned.

\section{Conclusion}
\label{intro:conclusion}

This chapter discusses the initial project idea \see{intro:overview}, aims \see{intro:aimsandobjectives} and any potential issues \see{intro:legal...issues}. These aspects were used to form a plan, influencing the literature review's focus and the artefact's development. Using the projects aims and subject outline, the following chapter forms the literature review, providing a greater understanding of the subject area at hand.